% !TEX program = ptex2pdf
%%		TO DO (by twice):
%%		ptex2pdf -l -u -ot "-kanji=utf8 "  -od "-p B5 " main
\RequirePackage{plext}			%	縱組顓用
\RequirePackage{plautopatch}	%	為 pLaTeX 打補丁
\RequirePackage{multicol} %多欄
\RequirePackage{type1cm} %字體
\RequirePackage[uplatex,deluxe,jis2004]{otf} %字體包

\documentclass[10pt,b5paper]{utbook}
\setlength{\textwidth}{192 mm}
\setlength{\textheight}{140 mm}
\usepackage{uline--}	% 输出波浪线、下划线等

% 居中标点 命名格式 \mxxx
\def\dou{\leavevmode\kern -.35zw \raise.6zh\hbox{\mc{・}}\kern -.15zw}
\def\dun{\leavevmode\kern -.35zw \raise.6zh\hbox{\mc{・}}\kern -.15zw}% 、
%\def\mgou{\leavevmode\kern -.35zw \raise.6zh\hbox{\mc{。}}\kern -.15zw}
% 非居中式标点
%\def\dou{\leavevmode\kern -.35zw \raise.3zh\hbox{\mc{,}}\kern -.15zw} %,
%\def\dun{\leavevmode\kern -.35zw \raise.3zh\hbox{\mc{、}}\kern -.15zw} % 、
\def\gou{\leavevmode\kern -.35zw \raise.3zh\hbox{\mc{。}}\kern -.15zw}
% 冒号、分号、叹号、问号替换为勾读。
\newcommand{\mao}{\dou}
\newcommand{\fen}{\gou}
\newcommand{\mtan}{\gou}
\newcommand{\wen}{\gou}
%前引号、后引号、书名号、曲牌名号。
\def\xqyh{\leavevmode\kern -.1zw\raise.1zh\hbox{\mc{『}}\kern-.1zw}
\def\xhyh{\leavevmode\kern-.05zw\raise-.1zh\hbox{\mc{』}}\kern -.1zw }
\def\qyh{\leavevmode\kern -.1zw\raise.1zh\hbox{\mc{「}}\kern-.1zw}
\def\hyh{\leavevmode\kern-.05zw\raise-.1zh\hbox{\mc{」}}\kern -.1zw }
\def\qsm{\leavevmode\kern -.1zw\hbox{\mc{〈}}}
\def\hsm{\leavevmode\hbox{\mc{〉}}\kern -.1zw}
\def\qqp{\leavevmode\kern -.1zw\hbox{\mc{〔}}}
\def\hqp{\leavevmode\hbox{\mc{〕}}\kern -.1zw}

\begin{document}
%%% 此處必須使用自定義的字體進行初始化,否則會產生蜜汁錯誤
\fontsize{15}{25}\selectfont
\mcfamily \pagestyle{empty}

\vspace*{8mm}
% 前言
{\LARGE\bfseries~題卓老批點西廂記}
\addcontentsline{toc}{chapter}{題卓老批點西廂記}

\vspace*{8mm}

\par\noindent
看書不從生動處看\dun
不從關鍵處看\dun
不從照應處看\dun
猶如相人不以骨氣\dun
不以神色\dun
不以眉目\dun
雖指點之工\dun
言驗之切\dun
下焉者矣\gou
烏得名相\dun
語曰傳神在阿堵間\dun
嗚呼\mtan
此處著眼\dun
正不易易\dun
吾獨怪夫世之耳\gou
食者\dun
不辯眞贋\dun
但借名色\dun
便爾稱隹\dun
如假卓老\dun
假文長\dun
假眉公\dun
種種諸刻\dun
盛行不諱\gou
及覩眞本\dun
反生疑詫\dun
掩我心靈\dun
隨人嗔喜\dun
舉世已盡然矣\gou
吾亦奚辯\dun
往陶不退語\gou
余家藏卓老西廂\dun
爲世所未見\gou
因舉\xqyh
風流隨何\dun
浪子陸賈\xhyh
二語\dun
疊用照應\dun
呼吸生動\dun
乃評之曰\dun
一用妙\dun
二用妙妙\dun
三用以致五用\dun
皆稱妙絕趣絕\gou
又如用頭巾語\dun
甚趣\gou
帶酸腐氣\dun
可愛\gou
往往點出\dun
皆人所絕不著意者\gou
一經道破\dun
煞有關情\gou
在彼作者\dun
亦不知技之至此極也\gou
卓老嘗言\dun
凡我批點\dun
如長康點睛\dun
他人不能代\gou
識此而後知卓老之書\dun
無有不切中關鍵\dun
開闊心胸\dun
發我慧性者矣\gou
夫西廂爲千古傳奇之祖\gou
卓老所批\dun
又爲西廂傳神之祖\gou
世不乏具眼\dun
應有取證在母曰劇本也\gou
當從李氏之書讀之矣\gou

\hfill{崇禎歲庚辰仲秋之朔酔香主人於快閣}\hspace*{2zw}

\hfill{〔酔香主人〕識}\hspace*{2zw}\\


\clearpage
\fontsize{15}{25}\selectfont
\mcfamily

\begin{quotation}
\par{}春遊浩蕩\hskip8pt是年年寒食\hskip8pt梨花時節\hskip8pt白錦無紋香爛漫\hskip8pt玉樹瓊苞堆雪\hskip8pt靜夜沉沉\hskip8pt浮光靄靄\hskip8pt冷浸溶溶月\hskip8pt人間天上\hskip8pt爛銀霞照通徹\hskip8pt渾似姑射眞人\hskip8pt天姿靈秀\hskip8pt意氣殊高潔\hskip8pt萬蕊參差誰信道\hskip8pt不與群芳同列\hskip8pt浩氣清爽\hskip8pt仙才卓犖\hskip8pt下土難分别\hskip8pt瑤天歸去\hskip8pt洞天方看清絶
\end{quotation}

看官\dou{}作這一首\hskip1mm\owave{無俗念}\hskip1mm詞的\dou{}乃是\kasen{南宋}末年一位武學名家\dou{}有道之士\gou{}此人姓\kasen{丘}\dou{}名\kasen{處機}\dou{}道號\kasen{長}\\
\kasen{春子}\dou{}名列\kasen{全眞七子}之一\dou{}是\kasen{全眞教}中出類拔萃的人物\gou\hskip1mm\owave{詞品}\hskip1mm評論此詞道\mao\qyh{}\kasen{長春}\dou{}世之所謂仙人也\dou{}而詞之清拔如此\gou\hyh{}這首詞誦的似是梨花\dou{}其實詞中眞意却是讚譽一位身穿白衣的美貌少女\dou{}説她\qyh{}渾似\kasen{姑射眞人}\dou{}天姿靈秀\dou{}意氣殊高潔\hyh\dou{}又説她\qyh{}浩氣清英\dou{}仙才卓犖\hyh\qyh{}不與群芳同列\hyh\gou{}詞中所誦這美女是誰\wen{}乃是\kasen{古墓派}傳人\kasen{小龍女}\gou{}她一生愛穿白衣\dou{}當眞如玉樹臨風\dou{}瓊苞堆雪\dou{}兼之生性清冷\dou{}實是當得起\qyh{}冷浸溶溶月\hyh{}的形容\dou{}以\qyh{}無俗念\hyh{}三字贈之\dou{}可説最妙不過\gou{}\kasen{長春子}\hskip1mm\kasen{丘處機}和她在\kasen{終南山}上比鄰而居\dou{}當年一見\dou{}便冩下這首詞來\gou

這時\kasen{丘處機}逝世已久\dou{}而\kasen{小龍女}也已嫁與\kasen{神鵰大俠}\hskip1mm\kasen{楊過}爲妻\gou{}可是在\kasen{河南}\hskip1mm\kasen{少室山}的山道之上\dou{}却另有一位少女\dou{}正在低低念誦此詞\gou{}這少女約有十八九歳年紀\dou{}身穿淡黃衣衫\dou{}騎著一頭瘦瘦的青驢\dou{}正沿著山道緩緩而上\gou{}她心中默想\mao\qyh{}也只有龍姊姊這樣的人物\dou{}纔配得上他\gou\hyh{}這一個\qyh{}他\hyh{}字\dou{}指的自然是\kasen{神鵰大俠}\hskip1mm\kasen{楊過}了\gou{}她也不拉韁\dou{}任著那青驢信步而行\dou{}一路上山\gou{}過了良久\dou\\
她又低聲吟道\mao\qyh{}歡樂趣\dou{}離别苦\dou{}就中更有痴児女\gou{}君應有語\dou{}渺萬里層雲\dou{}千山暮雪\dou{}隻影向誰去\wen\hyh



\end{document}

